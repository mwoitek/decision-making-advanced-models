% Created 2024-07-03 Wed 20:07
% Intended LaTeX compiler: pdflatex
\documentclass[11pt]{article}
\usepackage[utf8]{inputenc}
\usepackage[T1]{fontenc}
\usepackage{graphicx}
\usepackage{longtable}
\usepackage{wrapfig}
\usepackage{rotating}
\usepackage[normalem]{ulem}
\usepackage{amsmath}
\usepackage{amssymb}
\usepackage{capt-of}
\usepackage{hyperref}
\usepackage[a4paper,left=1cm,right=1cm,top=1cm,bottom=1cm]{geometry}
\usepackage[american, english]{babel}
\usepackage{enumitem}
\usepackage{float}
\usepackage[sc]{mathpazo}
\linespread{1.05}
\renewcommand{\labelitemi}{$\rhd$}
\setlength\parindent{0pt}
\setlist[itemize]{leftmargin=*}
\setlist{nosep}
\date{}
\title{Quiz: A Car Manufacturing Decision}
\hypersetup{
 pdfauthor={Marcio Woitek},
 pdftitle={Quiz: A Car Manufacturing Decision},
 pdfkeywords={},
 pdfsubject={},
 pdfcreator={Emacs 29.4 (Org mode 9.8)}, 
 pdflang={English}}
\begin{document}

\thispagestyle{empty}
\pagestyle{empty}
\section*{Problem 1}
\label{sec:org63c7681}

\textbf{Answer:} 10\\

First of all, I believe this question is not well formulated. As we'll see, the
mathematical problem we actually solve has 4 variables. To reach this
conclusion, I could have introduced as many auxiliary variables as I wanted or
none at all. So there is no unique answer to this question. All I can say is
that at least 4 variables are required. Then the only option that makes sense
is 10.
\section*{Problem 2}
\label{sec:orgaf27610}

\textbf{Answer:} \(400S_1+600S_2+150\mathrm{IC}_1+150\mathrm{IT}_1+150\mathrm{IC}_2+150\mathrm{IT}_2\)\\

In this problem, the cost has two components. First, there's the cost of the
steel needed to manufacture the vehicles. Then there's the holding cost.\\
We begin by computing the steel cost. To meet the demand, \(C_1\) cars are
manufactured during month 1, and \(C_2\) cars are manufactured during month 2.
Since every car requires 1 ton of steel, \(C_1\) and \(C_2\) are also the
number of tons of steel that need to be purchased during months 1 and 2,
respectively. This means that \(400C_1\) will be spent in the first month, and
\(600C_2\) will be spent in the second month. Therefore, the steel cost
corresponding to the cars is \(400C_1+600C_2\).\\
A similar argument can be used to determine the steel cost related to the
trucks. The only difference is that a truck requires 2 tons of steel. Then we
need \(2T_1\) tons for month 1, and \(2T_2\) tons for month 2. This leads to
a total steel cost of \(800T_1+1200T_2\). By adding this cost to the one
associated with the cars, we obtain the total steel cost:
\begin{equation}
C_s=400C_1+600C_2+800T_1+1200T_2.
\end{equation}
We can rewrite this expression in terms of \(S_1\) and \(S_2\) as follows:
\begin{align}
  \begin{split}
    C_s&=400C_1+600C_2+800T_1+1200T_2\\
    &=400C_1+800T_1+600C_2+1200T_2\\
    &=400(C_1+2T_1)+600(C_2+2T_2)\\
    &=400S_1+600S_2,
  \end{split}
\end{align}
where we've used \(S_1=C_1+2T_1\) and \(S_2=C_2+2T_2\).\\
Next, consider the holding cost. At the end of month 1, the number of vehicles
in inventory is \(\mathrm{IC}_1+\mathrm{IT}_1\). The corresponding holding
cost is \(150(\mathrm{IC}_1+\mathrm{IT}_1)\). Similarly, for month 2, there's
a cost of \(150(\mathrm{IC}_2+\mathrm{IT}_2)\). Therefore, the total holding
cost can be written as
\begin{align}
  \begin{split}
    C_h&=150(\mathrm{IC}_1+\mathrm{IT}_1)+150(\mathrm{IC}_2+\mathrm{IT}_2)\\
    &=150\mathrm{IC}_1+150\mathrm{IT}_1+150\mathrm{IC}_2+150\mathrm{IT}_2.
  \end{split}
\end{align}
Finally, we can write down the objective function. It is given by the sum of the
two components \(C_s\) and \(C_h\):
\begin{align}
  \begin{split}
    Z&=C_s+C_h\\
    &=400S_1+600S_2+150\mathrm{IC}_1+150\mathrm{IT}_1+150\mathrm{IC}_2+150\mathrm{IT}_2.
  \end{split}
\end{align}
\section*{Problem 3}
\label{sec:org79f90ea}

\textbf{Answer:} \(C_1+T_1\leq 1000\), \(C_2+T_2\leq 1000\)\\

At most 1,000 vehicles can be produced every month. During the first month,
\(C_1\) cars and \(T_1\) trucks are manufactured. Then, for month 1, we have
the constraint
\begin{equation}
C_1+T_1\leq 1000.
\end{equation}
Similarly, for the second month, we have
\begin{equation}
C_2+T_2\leq 1000.
\end{equation}
\section*{Problem 4}
\label{sec:org8e8c157}

\textbf{Answer:} \(C_1+2T_1=S_1\), \(C_2+2T_2=S_2\)\\

The reasoning behind this answer is in the solution to Problem 2.
\section*{Problem 5}
\label{sec:org68d8c2e}

\textbf{Answer:} \(200+C_1=800+\mathrm{IC}_1\)\\

At the end of month 1, there will be \(200+C_1\) cars (200 cars in inventory +
cars produced). 800 of those cars are for meeting the demand for that month. The
rest will be put in the inventory. Therefore, the following must hold:
\begin{equation}
200+C_1=800+\mathrm{IC}_1.
\end{equation}
\section*{Problem 6}
\label{sec:org77d3ac5}

\textbf{Answer:} \(4C_1-6T_1\geq 0\)\\

Recall that each car gets 20 mpg, and each truck gets 10 mpg. Then, for the
first month, the total mpg can be written as \(20C_1+10T_1\). Moreover, during
that month, \(C_1+T_1\) vehicles are manufactured. This allows us to express
the average mpg as
\begin{equation*}
\frac{20C_1+10T_1}{C_1+T_1}.
\end{equation*}
This average is required to be at least 16. Hence:
\begin{align}
  \begin{split}
    \frac{20C_1+10T_1}{C_1+T_1}&\geq 16\\
    20C_1+10T_1&\geq 16(C_1+T_1)\\
    20C_1+10T_1&\geq 16C_1+16T_1\\
    4C_1-6T_1&\geq 0.
  \end{split}
\end{align}
\section*{Problem 7}
\label{sec:org2c7a048}

\textbf{Answer:} \(S_1\leq 1500\), \(S_2\leq 1500\)
\section*{Problem 8}
\label{sec:org784518f}

\textbf{Answer:} Purchase 1400 tons of steel in month 1 and 700 tons in month 2.\\

Before we try to find the optimal solution, let us finish the mathematical
formulation of this problem. We need to reduce the number of variables we're
dealing with. It seems convenient to keep only the variables \(C_1\), \(T_1\),
\(C_2\) and \(T_2\). We already know how to use these variables to obtain
\(S_1\) and \(S_2\). Next, we derive similar results for the other auxiliary
variables:
\begin{itemize}
\item \(\mathrm{IC}_1\):
\begin{align}
  \begin{split}
    200+C_1&=800+\mathrm{IC}_1\\
    C_1&=600+\mathrm{IC}_1\\
    \mathrm{IC}_1&=C_1-600
  \end{split}
\end{align}
\item \(\mathrm{IT}_1\):
\begin{align}
  \begin{split}
    100+T_1&=400+\mathrm{IT}_1\\
    T_1&=300+\mathrm{IT}_1\\
    \mathrm{IT}_1&=T_1-300
  \end{split}
\end{align}
\item \(\mathrm{IC}_2\):
\begin{align}
  \begin{split}
    \mathrm{IC}_1+C_2&=300+\mathrm{IC}_2\\
    C_1-600+C_2&=300+\mathrm{IC}_2\\
    \mathrm{IC}_2&=C_1+C_2-900
  \end{split}
\end{align}
\item \(\mathrm{IT}_2\):
\begin{align}
  \begin{split}
    \mathrm{IT}_1+T_2&=300+\mathrm{IT}_2\\
    T_1-300+T_2&=300+\mathrm{IT}_2\\
    \mathrm{IT}_2&=T_1+T_2-600
  \end{split}
\end{align}
\end{itemize}
The next step is to rewrite the objective function. We already have a convenient
formula for \(C_s\). We only need to derive a new expression for \(C_h\). By
using the above equations, we get
\begin{align}
  \begin{split}
    C_h&=150(\mathrm{IC}_1+\mathrm{IT}_1+\mathrm{IC}_2+\mathrm{IT}_2)\\
    &=150(C_1-600+T_1-300+C_1+C_2-900+T_1+T_2-600)\\
    &=150(2C_1+2T_1+C_2+T_2-2400)\\
    &=300C_1+300T_1+150C_2+150T_2-360000.
  \end{split}
\end{align}
With the aid of this result, we can express \(Z\) as follows:
\begin{align}
  \begin{split}
    Z&=400C_1+600C_2+800T_1+1200T_2+300C_1+300T_1+150C_2+150T_2-360000\\
    &=700C_1+750C_2+1100T_1+1350T_2-360000.
  \end{split}
\end{align}
We want to minimize this function subject to the following constraints:
\begin{align}
  \begin{split}
    C_1&\geq 600\\
    T_1&\geq 300\\
    C_1+C_2&\geq 900\\
    T_1+T_2&\geq 600\\
    C_1+T_1&\leq 1000\\
    C_2+T_2&\leq 1000\\
    C_1+2T_1&\leq 1500\\
    C_2+2T_2&\leq 1500\\
    2C_1-3T_1&\geq 0\\
    2C_2-3T_2&\geq 0
  \end{split}
\end{align}
The first four inequalities guarantee that the demand for each month will be
met. The fifth and sixth constraints guarantee that no more than 1,000 vehicles
will be produced every month. The next pair of inequalities guarantee that at
most 1,500 tons of steel will be purchased each month. The last two constraints
correspond to the requirement for the average mpg.\\
We solve this optimization problem in a separate Python script.
\section*{Problem 9}
\label{sec:org1c82937}

\textbf{Answer:} Produce 600 cars and 400 trucks in month 1, and produce 300 cars and
200 trucks in month 2.
\section*{Problem 10}
\label{sec:orgacf3879}

\textbf{Answer:} \$995,000\\

This answer represents the minimum cost. There isn't enough information to
compute profit.
\end{document}
