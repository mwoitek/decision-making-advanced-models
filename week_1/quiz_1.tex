% Created 2024-07-03 Wed 01:57
% Intended LaTeX compiler: pdflatex
\documentclass[11pt]{article}
\usepackage[utf8]{inputenc}
\usepackage[T1]{fontenc}
\usepackage{graphicx}
\usepackage{longtable}
\usepackage{wrapfig}
\usepackage{rotating}
\usepackage[normalem]{ulem}
\usepackage{amsmath}
\usepackage{amssymb}
\usepackage{capt-of}
\usepackage{hyperref}
\usepackage[a4paper,left=1cm,right=1cm,top=1cm,bottom=1cm]{geometry}
\usepackage[american, english]{babel}
\usepackage{enumitem}
\usepackage{float}
\usepackage[sc]{mathpazo}
\linespread{1.05}
\renewcommand{\labelitemi}{$\rhd$}
\setlength\parindent{0pt}
\setlist[itemize]{leftmargin=*}
\setlist{nosep}
\date{}
\title{Banking Loan Decision}
\hypersetup{
 pdfauthor={Marcio Woitek},
 pdftitle={Banking Loan Decision},
 pdfkeywords={},
 pdfsubject={},
 pdfcreator={Emacs 29.4 (Org mode 9.8)}, 
 pdflang={English}}
\begin{document}

\thispagestyle{empty}
\pagestyle{empty}
\section*{Problem 1}
\label{sec:org31c63f0}

\textbf{Answer:} 5\\

There are five decision variables, one for each type of loan.
\section*{Problem 2}
\label{sec:orge0bffc5}

\textbf{Answer:} \(0.026x_1\)\\

Let's denote the expected net return by \(R_1\). This value can be expressed as
\begin{equation}
R_1=r_1 x_1(1-p_1)-x_1 p_1,
\end{equation}
where \(r_1\) represents the interest rate, and \(p_1\) is the probability
of bad debt. By substituting the values of these quantities into the last
equation, we get
\begin{align}
  \begin{split}
    R_1&=r_1 x_1(1-p_1)-x_1 p_1\\
    &=0.14(1-0.1)x_1-0.1x_1\\
    &=0.126x_1-0.1x_1\\
    &=0.026x_1.
  \end{split}
\end{align}
\section*{Problem 3}
\label{sec:orgd74a25f}

\textbf{Answer:} \(0.0509x_2\)\\

Let's denote the expected net return by \(R_2\). This value can be expressed as
\begin{equation}
R_2=r_2 x_2(1-p_2)-x_2 p_2,
\end{equation}
where \(r_2\) represents the interest rate, and \(p_2\) is the probability
of bad debt. By substituting the values of these quantities into the last
equation, we get
\begin{align}
  \begin{split}
    R_2&=r_2 x_2(1-p_2)-x_2 p_2\\
    &=0.13(1-0.07)x_2-0.07x_2\\
    &=0.1209x_2-0.07x_2\\
    &=0.0509x_2.
  \end{split}
\end{align}
\section*{Problem 4}
\label{sec:org209078e}

\textbf{Answer:}
\begin{equation*}
\max\quad Z=0.026x_1+0.0509x_2+0.0864x_3+0.06875x_4+0.078x_5
\end{equation*}
\vspace{0.1cm}

To derive the formula for the objective function, we need to perform
calculations similar to the ones in Problems 2 and 3. Specifically, we have to
compute the expected net return for the three remaining types of loan. We won't
present the details here. However, it's straightforward to show that the
expected returns associated with Home, Farm and Commercial loans can be
expressed as follows:
\begin{itemize}
\item Home: \(0.0864x_3\)
\item Farm: \(0.06875x_4\)
\item Commercial: \(0.078x_5\)
\end{itemize}
Then the total expected return is
\begin{equation}
Z=0.026x_1+0.0509x_2+0.0864x_3+0.06875x_4+0.078x_5.
\end{equation}
It should be clear that the bank wants to maximize its return. Therefore, we
need to find \(\max Z\).
\section*{Problem 5}
\label{sec:org612217b}

\textbf{Answer:} \(x_1+x_2+x_3+x_4+x_5\leq 10\)\\

The total amount the bank will spend in loans is \(x_1+x_2+x_3+x_4+x_5\). At
most, the total spent will be \(\$ 10\) million. Since the unit of all \(x_i\)
is \(\$ 1\) million, we can write this constraint as follows:
\begin{equation}
x_1+x_2+x_3+x_4+x_5\leq 10.
\end{equation}
\section*{Problem 6}
\label{sec:org8f7664c}

\textbf{Answer:} \(0.3x_1+0.3x_2+0.3x_3-0.7x_4-0.7x_5\leq 0\)\\

The amount allocated to Farm and Commercial loans is \(x_4+x_5\). This amount
should be at least
\begin{equation*}
0.3\left(x_1+x_2+x_3+x_4+x_5\right).
\end{equation*}
Therefore, this constraint can be written as
\begin{align}
  \begin{split}
    x_4+x_5&\geq 0.3\left(x_1+x_2+x_3+x_4+x_5\right)\\
    x_4+x_5&\geq 0.3x_1+0.3x_2+0.3x_3+0.3x_4+0.3x_5\\
    0.3x_1+0.3x_2+0.3x_3+0.3x_4+0.3x_5&\leq x_4+x_5\\
    0.3x_1+0.3x_2+0.3x_3-0.7x_4-0.7x_5&\leq 0
  \end{split}
\end{align}
\section*{Problem 7}
\label{sec:orgaba22a5}

\textbf{Answer:} \(0.4x_1+0.4x_2-0.6x_3\leq 0\)\\

Recall that the amount spent in Home loans is \(x_3\). The amount allocated to
Personal, Car and Home loans is given by \(x_1+x_2+x_3\). Since Home loans
must equal at least 40\% of this sum, the following must hold:
\begin{align}
  \begin{split}
    x_3&\geq 0.4\left(x_1+x_2+x_3\right)\\
    x_3&\geq 0.4x_1+0.4x_2+0.4x_3\\
    0.4x_1+0.4x_2+0.4x_3&\leq x_3\\
    0.4x_1+0.4x_2-0.6x_3&\leq 0
  \end{split}
\end{align}
\section*{Problem 8}
\label{sec:org80fa2fe}

\textbf{Answer:} \(0.06x_1+0.03x_2-0.01x_3+0.01x_4-0.02x_5\leq 0\)\\

For the amount \(x_i\), we expect the bank to lose \(x_i p_i\) due to bad
debt. Then the total loss can be expressed as \(x_1 p_1+x_2 p_2+x_3 p_3+x_4 p_4+x_5 p_5\).
By using the probability values, we can rewrite this loss: \(0.1x_1+0.07x_2+0.03x_3+0.05x_4+0.02x_5\).
The overall ratio of bad debts corresponds to the total loss divided by the
total spent. Since this ratio is not to exceed 0.04, we have
\begin{align}
  \begin{split}
    \frac{0.1x_1+0.07x_2+0.03x_3+0.05x_4+0.02x_5}{x_1+x_2+x_3+x_4+x_5}&\leq 0.04\\
    0.1x_1+0.07x_2+0.03x_3+0.05x_4+0.02x_5&\leq 0.04x_1+0.04x_2+0.04x_3+0.04x_4+0.04x_5\\
    0.06x_1+0.03x_2-0.01x_3+0.01x_4-0.02x_5&\leq 0
  \end{split}
\end{align}
\section*{Problem 9}
\label{sec:org7ef28a5}

\textbf{Answer:} \(\$ 0.8388\) million
\section*{Problem 10}
\label{sec:orga403908}

\textbf{Answer:} 2
\section*{Problem 11}
\label{sec:orgbb6ad80}

\textbf{Answer:} \(\$ 7\) million in home loans and \(\$ 3\) million in commercial loans
\end{document}
