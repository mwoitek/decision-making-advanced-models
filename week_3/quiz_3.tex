% Created 2024-07-04 Thu 22:08
% Intended LaTeX compiler: pdflatex
\documentclass[11pt]{article}
\usepackage[utf8]{inputenc}
\usepackage[T1]{fontenc}
\usepackage{graphicx}
\usepackage{longtable}
\usepackage{wrapfig}
\usepackage{rotating}
\usepackage[normalem]{ulem}
\usepackage{amsmath}
\usepackage{amssymb}
\usepackage{capt-of}
\usepackage{hyperref}
\usepackage[a4paper,left=1cm,right=1cm,top=1cm,bottom=1cm]{geometry}
\usepackage[american, english]{babel}
\usepackage{enumitem}
\usepackage{float}
\usepackage[sc]{mathpazo}
\linespread{1.05}
\renewcommand{\labelitemi}{$\rhd$}
\setlength\parindent{0pt}
\setlist[itemize]{leftmargin=*}
\setlist{nosep}
\date{}
\title{Quiz: An Airline Pilot Staffing Decision}
\hypersetup{
 pdfauthor={Marcio Woitek},
 pdftitle={Quiz: An Airline Pilot Staffing Decision},
 pdfkeywords={},
 pdfsubject={},
 pdfcreator={Emacs 29.4 (Org mode 9.8)}, 
 pdflang={English}}
\begin{document}

\thispagestyle{empty}
\pagestyle{empty}
\section*{Problem 1}
\label{sec:org3690d84}

\textbf{Answer:} 12\\

For every month, we need two decision variables: the number of pilots who are
hired, and the number of pilots who are fired. Since we're considering 6
months, there are 12 decision variables.
\section*{Problem 2}
\label{sec:org1a210ee}

\textbf{Answer:} \(120+x_{H_1}-x_{F_1}\)\\

At the beginning of the first month, the company has 120 pilots. \(x_{H_1}\)
pilots are hired, and \(x_{F_1}\) pilots are fired. Therefore, the number of
available pilots is
\begin{equation*}
120+x_{H_1}-x_{F_1}.
\end{equation*}
\section*{Problem 3}
\label{sec:org751edc1}

\textbf{Answer:} \(120+x_{H_1}-x_{F_1}\geq 100\)\\

In month 1, the company needs at least 100 pilots. We've just derived an
expression for the number of available pilots. By using this result, we can
write
\begin{equation}
120+x_{H_1}-x_{F_1}\geq 100.
\end{equation}
\section*{Problem 4}
\label{sec:orga864a00}

\textbf{Answer:} \texttt{D5:I6}
\section*{Problem 5}
\label{sec:org8821e3b}

\textbf{Answer:} \(4(120+x_{H_1}-x_{F_1})+5x_{H_1}+3x_{F_1}\)\\

To write down a formula for this cost, we'll express all the amounts that
represent money in units of \$1,000. Every pilot who is still a staff member has
to be paid his salary of \$4k. The number of such pilots has been computed in
Problem 2. By using that result, we can conclude that the cost related to
salaries is \(4(120+x_{H_1}-x_{F_1})\). Furthermore, each of the \(x_{H_1}\)
pilots who were hired represents a cost of \$5k. Then the corresponding cost is
\(5x_{H_1}\). Finally, there's the cost of \$3k for every pilot who were fired.
In total, this cost is \(3x_{F_1}\). By adding it all up, we get
\begin{equation*}
4(120+x_{H_1}-x_{F_1})+5x_{H_1}+3x_{F_1}.
\end{equation*}
\section*{Problem 6}
\label{sec:org5ae45bb}

\textbf{Answer:} \texttt{C11+D5-D6}
\section*{Problem 7}
\label{sec:org60510a7}

\textbf{Answer:} \texttt{D11*\$C\$13+D5*\$C\$8+D6*\$C\$9}
\section*{Problem 8}
\label{sec:org38ae723}

\textbf{Answer:} \$3740\\

To answer this question and the ones that follow, first we need to complete the
mathematical formulation of this problem. We already know what happens during
the first month. So consider month 2 next. At the beginning of this month, there
are \(N_1=120+x_{H_1}-x_{F_1}\) pilots working for the company. \(x_{F_2}\)
of them will be fired, and \(x_{H_2}\) new pilots will be hired. Then, in
month 2, there will be \(N_2=N_1+x_{H_2}-x_{F_2}\) available pilots.\\
As we have done for month 1, we can derive the constraint for \(N_2\) and
compute the corresponding contribution to the cost. To obtain the constraint, we
can do something very similar to what we've done in Problem 3. What we get is
\begin{equation}
N_1+x_{H_2}-x_{F_2}\geq 110.
\end{equation}
The cost for the second month can be calculated as we've done in Problem 5. Very
similar reasoning allows us to express this cost as follows:
\begin{equation}
4(N_1+x_{H_2}-x_{F_2})+5x_{H_2}+3x_{F_2}.
\end{equation}
We could repeat this process four more times to get the desired results for the
remaining months. We won't present the details, and simply write down these
results. First consider the number \(N_i\) of pilots available in month \(i\):
\begin{align}
  \begin{split}
    N_1&=120+x_{H_1}-x_{F_1},\\
    N_2&=N_1+x_{H_2}-x_{F_2},\\
    N_3&=N_2+x_{H_3}-x_{F_3},\\
    N_4&=N_3+x_{H_4}-x_{F_4},\\
    N_5&=N_4+x_{H_5}-x_{F_5},\\
    N_6&=N_5+x_{H_6}-x_{F_6}.
  \end{split}
\end{align}
By introducing \(N_0=120\), we can summarize this set of equations as follows:
\begin{equation}
N_i=N_{i-1}+x_{H_i}-x_{F_i},\qquad i=1,\ldots,6.
\end{equation}
There's a constraint for each of these numbers. These constraints guarantee
that, during months 1-6, the company has enough pilots. These inequalities are
the following:
\begin{align}
  \begin{split}
    N_1&\geq 100,\\
    N_2&\geq 110,\\
    N_3&\geq 115,\\
    N_4&\geq 140,\\
    N_5&\geq 110,\\
    N_6&\geq 200.
  \end{split}
\end{align}
Finally, consider the cost \(C_i\) for the \(i\)-th month. This amount can
be written as
\begin{equation}
C_i=4N_i+5x_{H_i}+3x_{F_i},\qquad i=1,\ldots,6.
\end{equation}
The objective function for this problem corresponds to the total cost. The total
is obtained from adding up all the \(C_i\):
\begin{equation}
Z=\sum_{i=1}^6 C_i=\sum_{i=1}^6(4N_i+5x_{H_i}+3x_{F_i}).
\end{equation}
We need to minimize this function subject to the 6 constraints listed above, and
to non-negativity constraints for all the decision variables \(x_{H_i}\) and
\(x_{F_i}\).
\section*{Problem 9}
\label{sec:orgb350e80}

\textbf{Answer:} 3
\section*{Problem 10}
\label{sec:org0f0c806}

\textbf{Answer:} 25
\end{document}
